% !TeX_ROOT=../thesis.tex

\chapter{TFHE}



\section{Basics on $\LWE$ and $\GLWE$ Encryption}


\subsection{Hardness assumptions}


\paragraph{Original $\LWE$ problem}

In 2005, Regev defined the Learning With Errors ($\LWE$) problem in \cite{regev_lwe}. With this work, he laid the foundations for an important part of modern lattice-based cryptography. The version usually used in FHE is presented in Definition \ref{def:LWE}:


\begin{definition}
	(Learning with Errors). Let $q$ and $n$ two integers, respectively called \textit{modulus} and \textit{dimension}.  and let $\chi_s$ and $\chi_e$ be distributions over the small values of $\lweRing$. We sample a vector $\vec s = (s_0, \dots, s_{n-1}) \drawfrom \chi_s^n$. 
	
	We define the $\LWE$ distribution $\mathcal{D}_{q, n, \chi_s, \chi_e}^{\LWE}$ as:
	
	 \[
	 \mathcal{D}_{q, n, \chi_s, \chi_e}^{\LWE} = \left \{(\vec a, b) \;\middle|\; \vec a = (a_0, \dots, a_{n-1}) \drawfrom \unif{\lweRing}^n, e \drawfrom \chi_e, b = \sum_{j=0}^{n-1} a_j \cdot s_j + e \right \}
	  \]
	 
	 The \textit{decisional} version of the problem is to distinguish this distribution from a uniformely random one, namely:
	
	\[
	\mathcal{D}^{(\textsf{random})} = \left \{(\vec a, r) \;\middle|\; \vec a \drawfrom \unif{\lweRing}^n, r \drawfrom \unif{\lweRing} \right\}
	\]

	The \emph{search} version of the problem is to recover $\vec s$ from samples of $\mathcal{D}_{q, n, \chi_s, \chi_e}^{\LWE}$. 
	\label{def:LWE}
\end{definition}


Regev showed that the search and decisional problems are reducible to each other and their average case is as hard as worst-case lattice problems.

The hardness of this problem depends on the parameters $q$, $n$, $\chi_s$ and $\chi_e$, and so does the security of the schemes built upon it. To derive a concrete security level $\lambda$ from a parameter set, state of the art is a tool named \texttt{lattice-estimator} \cite{lattice-estimator}. Users can input concrete values and distributions for the parameters, and the tool evaluates the security of the underlying $\LWE$ instance by running simulations of attacks of the literature.

In Definition \ref{def:LWE}, we did not precise the shapes of the distributions $\chi_s$ and $\chi_e$ (beyond yielding small values). So many distributions are possible: a discrete Gaussian with a small variance, a uniform distribution restricted on a small interval, or a binomial. 

Most of implementations of TFHE select a uniform distribution on $\{0, 1\}$ for the secret, and a Gaussian with a small variance $\lweSigma$. Thus, we will use these one in this thesis. We will use the notation $\LWE_{(q, n, \sigma)}$ for these instances.



\paragraph{Extension to the Polynomials: $\GLWE$}


Looking for more efficient solutions, $\LWE$ problem has been declined in a \textit{ring variant} in \cite{rlwe}, and further expanded in a multitude of variants since them. A generalized version over ring, named $\GLWE$ and used by TFHE, is presented below. It is very similar to the original one, but deals with polynomial values instead of integers:

\begin{definition}
	(Generalized Learning with Errors) Let $q$, $k$ and $N$ three integers, respectively called \textit{modulus}, \textit{dimension} and \textit{degree}. Let $\chi_S$ and $\chi_E$ be distributions over the small values of $\glweRing$. We sample a vector $\vec S = (S_0, \dots, S_{k-1}) \drawfrom \chi_S^k$. 
	
	We define the $\GLWE$ distribution $\mathcal{D}_{q, k, N, \chi_S, \chi_E}^{\GLWE}$ as:
	
	\[
	\mathcal{D}_{q, k, N, \chi_S, \chi_E}^{\GLWE} = \left \{ (\vec A, B) \;\middle|\; \vec A = (A_0, \dots, A_{k-1}) \drawfrom \unif{\glweRing}^k, E \drawfrom \chi_E, B = \sum_{j=0}^{k-1} A_j \cdot S_j + E \right \}
	\]
	
	The \textit{decisional} version of the problem is to distinguish this distribution from a uniformely random one, namely:
	
	\[
	\mathcal{D}^{(\textsf{random})} = \left \{(\vec A, R) \;\middle|\; \vec A \drawfrom \unif{\glweRing}^k, R \drawfrom \unif{\glweRing} \right\}
	\]
	
	The \emph{search} version of the problem is to recover $\vec S$ from samples of $\mathcal{D}_{q, k , N, \chi_S, \chi_E}^{\GLWE}$. 
	\label{def:GLWE}
	
\end{definition}

Note that if we fix $k = 1$, we fall back on the classical $\RLWE$ problem, notably used in BGV \cite{bgv}. Also, taking $N=1$ produces a $\LWE$ instance with $n = k$. 

Concretely, using polynomial rings allows to encode more information in a single sample, yielding more compact ciphertexts and public keys. The schemes can also benefit from high-speed polynomial arithmetic techniques such as FFT. 

General consensus is that hardness of an instance $\GLWE_{(q, k, N, \sigma)}$ is similar to the hardness of $\LWE_{(q, k \cdot N, \sigma)}$, which makes possible to use the \texttt{lattice-estimator} as well.


\TODO{Je ne trouve pas de reference pour ce claim, mais c'est ce que tout le monde fait}






\section{A Word on the Torus equivalence and its discretization}
\label{sec:torus_equivalence}


The T in TFHE stands for \textit{Torus}. This is because in the seminal paper of TFHE \cite{JC:CGGI23}, authors worked with a torus-based version of $\LWE$.


The torus $\T = \R / \Z$ corresponds to the integers modulo 1. It has a $\Z$-module structure, which means that :

\begin{itemize}
	\item The sum of two torus elements is well-defined.
	\item The multiplication between an element of $\T$ and an element of $\Z$ is also well-defined, and produces an element of $\T$.
	\item On the other hand, multiplying two elements of $\mathbb T$ does not make sense. To be convinced of it, we can remark that, for any non-zero torus element $x$, $0 \times x = 0$ while $1 \times x = x$. But since 0 and 1 are equivalent over the torus, these results should not be different. 
\end{itemize}



Recall the $\LWE$ assumption (Definition \ref{def:LWE}). If we rescale the elements of $\lweRing$ by dividing them by $q$, we get elements of the torus. We can then redefine seamlessly the $\LWE$ problem over the torus. Extensive details about this transformation can be found in \cite{these_chillotti}.


This brings two advantages:

\begin{itemize}
	\item $\LWE$ over the torus is \textit{scale-invariant}, which makes the analysis of the security and of the noise much simpler.
	\item The $\Z$-module structure propagate in the ring versions, as well in matrices spaces. Thus, it allows for very powerful generalizations of homomorphic schemes on a wide variety of spaces, like in \cite{chimera, asiacrypt}.
\end{itemize} 


When implementing the scheme in practice, torus elements are represented by integer types. The torus is thus seen as \textit{discretized}, which we denote by 

\[ \T_q = \left \{   \frac a q \;\middle|\; a \in \Z_q  \right \} \] 

with $q = 2^\Omega$ ($\Omega$ denotes the number of bits of precision of the concrete type, so often 32 or 64 bits in most implementations). The properties of the torus structure are preserved.


This thesis is mainly about practical instantiations of the scheme. So, for the sake of clarity we will adopt a notation closer to the reality of the objects manipulated in machine. So the torus elements will be seen as elements of $\lweRing$ (but keeping the algebraic rules imposed by the structure of $\T$), and the same will be applied for the ring extensions. 



\section{Encryption and Decryption in TFHE}


The plaintext space of TFHE is the \textit{discretized torus} $\plaintextTorus$. As explained in Section \ref{sec:torus_equivalence}, we trivially identify it to the ring $\plaintextRing$ with $p$ an integer. Let us consider a mapping $\rho: \plaintextRing \rightarrow \lweRing$ (in practice, $p \ll q$), defined as \[
\rho: m  \mapsto \rounding{\frac{m q} {p}}.
\]
The image of this mapping only reaches $p$ elements in $\lweRing$, forming the set $\left \{ \frac {k q}{p} \mid k \in \plaintextRing \right \}$. These elements are evenly distributed across $\lweRing$ and form what we refer to as \emph{sectors of $\lweRing$}, defined as: \[
\left \{ \left ( \frac{(2 k - 1)q}{2p}, \frac{(2k + 1)q}{2p} \right ) \mid k \in \plaintextRing \right \}.
\]

TFHE features two types of encryption that share similar structural patterns but operate within different mathematical spaces.


\paragraph{LWE Encryption.} Let $m \in \plaintextRing$ be a message and let $\lweSecretKey=(s_1, \dots, s_n)$ represent the secret key, sampled uniformly at random from $\B^n$. First, the message $m$ is encoded into the space $\lweRing$ by $\tilde m = \rho(m)$. A small random Gaussian noise $e \drawfrom \chi_{\lweSigma}$ is then added. Since $e$ is small, the noisy message $\tilde m + e$ remains within the same sector as $\tilde m$. Next, we construct the LWE ciphertext as a vector \[c = (a_0, \dots, a_{n-1}, b)\], where the $a_i$'s are sampled uniformly at random from $\lweRing$, and $b$ is defined by \[b=\sum_{i=0}^{n-1} a_i \cdot s_i + \tilde m + e.\]

Decryption is performed in two steps: first, we compute $\phi(c) = b - \sum_{i=0}^{n - 1} a_i \cdot s_i$, referred to as the \emph{phase} of the ciphertext. Then we round it to the nearest plaintext value: $\tilde m = \rounding{\frac p q \phi(c)}$. As long as $|e| < \frac{q}{2p}$, this rounding produces the right sector center, and thus we recover the correct plaintext value.

The security of this encryption relies on the hardness of the assumption $\LWE_{(q, n, \lweSigma)}$. 



\paragraph{GLWE Encryption.} This encryption mode mirrors the structure of $\LWE$ encryption but operates within polynomial rings. The secret key $\glweSecretKey$ is here represented as a vector $(S_0, \dots, S_{k-1})$, sampled uniformly at random from $\B_{N, q}[X]^k$. 
%
The message is encoded in a polynomial $\tilde M \in \glweRing$. The noise is also a polynomial from the same ring, whose coefficients are drawn from the distribution $\chi_{\glweSigma}$.
%
Similar to $\LWE$ encryption, the ciphertext takes the form \[C = (A_0, \dots, A_{k-1}, B)\] where the $A_i$'s are sampled uniformely at random in $\glweRing$ and $B$ is defined by: \[B = \sum_{i=0}^{k-1} A_i \cdot S_i + \tilde M + E.\]
%

Decryption follows the same steps as the $\LWE$ case: the phase is computed as $\phi(C) = \sum_{i=0}^{k-1} A_i \cdot S_i$ and rounded to the closest plaintext value.


The security of this encryption relies on the hardness of the assumption $\GLWE_{(q, k, N, \glweSigma)}$. Because the plaintext is a polynomial of degree $N$, it is possible to trivially batch up to $N$ plaintexts from $\plaintextRing$ by encoding them in the coefficients. With this method, they can be processed in parallel by the linear homomorphism presented below.




\section{Linear Homomorphisms}

On a torus, two operations are well-defined: the sum of two torus elements and the external product between a torus element and a \textit{scalar}.

It is not hard to see that both TFHE encryption modes are linearly homomorphic. We define the two operations \texttt{sumTFHE} and \texttt{clearMultTFHE}, that are well-defined for both encryption flavours:


\paragraph{$\sumTFHE{c}{c'}$:} Let $c = (a_0, \dots, a_{n-1}, b)$ and $c' = (a_0', \dots, a_{n-1}', b')$ be two $\LWE$ ciphertexts encrypting the messages $m$ and $m'$ from $\plaintextRing$, with respective noise variance $\sigma$ and $\sigma'$. Summing coefficient-wise both ciphertexts yields a new ciphertext $c'' = (a_0 + a_0', \dots, a_{n-1} + a'_{n-1}, b + b')$ encrypting $m + m'$ with a larger noise $e + e'$. The exact same algorithm works with polynomials for $\GLWE$ encryption. 


\paragraph{$\clearMultTFHE{c}{\lambda}$:} Let $c = (a_0, \dots, a_{n-1}, b)$ a $\LWE$ ciphertext encrypting a message $m \in \plaintextRing$ and $\lambda \in \plaintextRing$ a constant. Multypling each coefficient by the constant yields a new ciphertext $c' = (\lambda \cdot a_0, \dots, \lambda \cdot a_{n-1}, \lambda \cdot b)$ encrypting the message $\lambda \cdot m$ with a larger noise $\lambda \cdot e$.

The same algorithm works for polynomials. The constant can be either a scalar from $\plaintextRing$ or a polynomial itself.



%Actually, it may be useful to do it in Z_q blabla probl_me bruit blabla external product avec la décomposition

\TODO{Revérifier que je ne dis pas de conneries, et que faire de ce paragraphe ?}
We stress once again that the products between two elements of the torus is not well defined. Yet, it may be handy to be able to multiply two elements of $\plaintextRing$ regardless of the inherent torus structure of TFHE. To do so, it is necessary to \textit{lift} one of the terms to an external space (such as [...]) and run an external product. This requires to perform homomorphically the gadget decomposition introduced in Section \ref{external_products} which is not straightforward. This is called in the literature a \textit{circuit bootstrapping} (citer et faire le lien avec le packing ?)
\TODO{Vérifier dans la littérature si le terme \textit{Internal Product} est adapté, et pourquoi ce terme existe ?}


\section{Efficient External Products}







\section{Key Switching and Bootstrapping}







\section{A deep-dive into the Programmable Bootstrapping algorithm}

In his seminal paper, Gentry introduced the concept of bootstrapping, that can be summed up in one phrase by:

\begin{quote}
	Evaluating homomorphically the decryption circuit on a noisy ciphertext produces a new ciphertext encrypting the same value, but with a smaller noise level.
\end{quote}

So informally, for a bootstrapping procedure to work, one needs to provide the server \textit{an encryption of the secret key}. This key is called a \textit{Bootstrapping Key}.



Let $c$ be a ciphertext encrypting a message $m$, with some noise $e$, under a key $s$. We recall that TFHE's decryption circuit looks like:

\begin{enumerate}
	\item Computing the phase of the ciphertext: $\phi(c) = b - \sum_{i=0}^{n-1} a_i \cdot s_i = m + e$.
	\item Rounding the phase to the closest plaintext: $m = \rounding{\phi(c)}$.
\end{enumerate}


Step 1. being purely linear, it is very easy to perform using TFHE's trivial linear homomorphism (at the cost of significant noise growth). But Step 2. is less clear: How to perform rounding homomorphically ? 

In the following, we will explain these two steps, and introduce key concepts such as \textit{Gadget Decompositions}, \textit{External Products} or \textit{Blind Rotation}. A first piece will be dedicated to computing Step 1. with the noise growth the most limited possible. Then, we will see how to implement Step 2. by leveraging some polynomial algebra.

\section{Computing the scalar product without noise explosion}

Let $c = (a_0, \dots, a_{n-1}, b)$ be a \LWE ciphertext encrypting a message $m$ under a key $s = (s_0, \dots, s_{n-1}) \in \B^n$. The client has encrypted this key under a second one $s_2$. For now, we suppose that the \LWE flavour of encryption is used as well (we will see in the following that this is not the case in practice, but it makes the explanation clearer). So, the \textit{bootstrapping key} is a collection of $n$ ciphertexts encrypting each of the bits of $s$.

\begin{equation}
	\BSK = \left ( \texttt{Enc}(s_0), \dots, \texttt{Enc}(s_{n-1}) \right )
\end{equation}



From $c$ and $\BSK$, it is easy to see that Step 1. can be trivially computed homomorphically by:

\begin{equation*}
	\sumTFHETernary{\clearMultTFHE{\texttt{Enc}(s_0)}{a_0}}{\dots}{\clearMultTFHE{\texttt{Enc}(s_{n-1})}{a_{n-1}}}
\end{equation*}


The major problem with this approach lies in the \textit{noise growth}: remember that the $a_i$'s are sampled at random in the ring $\lweRing$. So the expectation of their magnitude will be very high relatively to the size of the ring. But when multiplying with a cleartext, the noise of a ciphertext is multiplied by the same amount. Thus, we need a more sophisticated multiplication algorithm with a better noise behavior.

Enters \textit{Gadget Decomposition}, introduced in \cite{GSW13}. We give an informal view of the algorithm in the following paragraph.


\paragraph{Gadget Decomposition: }


Let $c$ be a \LWE ciphertext and $\alpha$ a constant, sampled uniformely at random in $\lweRing$. Directly multiplying every component of the ciphertext by $\alpha$ would also multiply the noise by the same amount. Gadget Decomposition is a construction allowing to performm this homomorphic multiplication while limiting noise growth.





\section{A look at Blind Rotation}

\TODO{Trouver le papier d'où ça vient, et l'expliquer aved des zolis dessins}



\section{External product}


