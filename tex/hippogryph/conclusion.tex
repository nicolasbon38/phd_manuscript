\section{Conclusion}

Even if this paper focuses primarily on AES, it should be seen as a first step towards solving the boolean-vs-byte tension which often occurs when attempting to run algorithms over TFHE. Beyond the quest for ``the fastest AES-over-TFHE in the west'', this paper's approach will clearly benefit to other block-ciphers such as PRINCE \cite{PRINCE}, SKINNY \cite{skinny} or PRESENT \cite{present}, which also alternate boolean- and byte-friendly operations. For instance the 4-bit SBox of PRINCE is byte-friendly (and even more efficient with the full-LUT approach than the 8-bit Sbox of the AES). But the PRINCE matrix multiplication is a real efficiency bottleneck for this approach, as it only consists of XOR and AND operations on bits of the state matrix.

Furthermore, although AES may seem an arbitrary benchmark, it can however be expected that works on this algorithm prefigure more widely applicable advances. For instance, the work in \cite{DBLP:conf/wahc/TramaCBS23} later leads to the full-blown instruction set in \cite{d8fp} as a systematization of the LUT-based approach. An interesting perspective would then be to revisit that latter instruction set by taking advantage of the toolbox proposed in the present paper. %Lastly, other specific algorithms may benefit from our approach, for example when running a CNN with, for each neuron, the initial dot products more efficiently implemented by means of $p$-encodings and the activation function naturally performed by means of the LUT-based approach. \issue{This last sentence is a sketch, I forgot why $p$-encodings are better for dot products, because we assume binary inputs to the CNN neurons? If so, we need a citation. We may remove this last sentence if unsure}.
%\NB{Finalement je ne suis pas sur que le CNN soit un si bon exemple que ca, car il y a des bootstrappings reguliers pour cleaner le bit de padding.} Je vais essayer de reflechir a d'autres exemples

%\issue{Lastly, we plan to open source the unified software test bench for AES execution over TFHE we created for obtaining the consistent same-machine experimental results given in this paper (provided so far in supplementary material).} 


%Emphasis applicability beyond AES (perhaps talk about doing a CNN with LUT-based activation function and $p$-encodings based dot products).

%Dans les perspectives, évoquer de façon quantitavive l'efficacité de la méthode sur d'autres block ciphers comme PRINCE, SKINNY ou PRESENT.


%Emphasis sur test bench pour les travaux futurs sur l'AES.