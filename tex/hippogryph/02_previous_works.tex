
\section{Some Building Blocks for LUT-based Evaluation}
\label{sec:previous-blocks}


In this section, we present the approach from \cite{DBLP:conf/wahc/TramaCBS23}, some components of which we use for our own work. We formally present some advanced homomorphic primitives used in this work that we reuse as well.


\cite{DBLP:conf/wahc/TramaCBS23} is a ``Full-\acrshort{LUT}'' approach, that is to say \gls{AES} is evaluated entirely with \gls{TFHE}'s programmable bootstrapping, encoding exclusively all operations within LUTs. To meet the performance constraints of the bootstrapping algorithm, this method operates on elements in $\Z_{16}$, ensuring efficient computation.

\subsection{AES Subroutines as LUTs}

The \SubBytes step, which involves the evaluation of an \gls{S-box}, is inherently a \gls{LUT} operation and is therefore naturally implemented in \gls{FHE} using a \gls{PBS}. However, \gls{PBS} is too slow in $\F_{256}$ (as we have seen in Section \ref{sec:pbs_performances}). So, they rely on a construction evaluating \gls{PBS} over $\Z_{16}$ rather than $\F_{256}$. Moreover, converting the other \gls{AES} steps into \gls{LUT} evaluations also requires additional effort.

In particular, in the original \gls{AES} design \cite{aes-original}, the \MixColumns step is computed using a series of XOR operations and multiplications in $\F_{256}$. Unfortunately, \gls{TFHE}’s native multiplication \texttt{ClearMultTFHE} cannot directly handle these $\F_{256}$ multiplications because of the polynomial nature of the elements of this field. As a result, \MixColumns must be reformulated as a \gls{LUT} evaluation.

Additionally, the \AddRoundKey step, which uses XOR as its key operation, presents its own challenges because XOR is a bivariate operation that requires two inputs. Classical bootstrapping, which operates on single inputs, is insufficient for this purpose. To address this, the authors utilize a specialized bootstrapping method that supports operations on multiple encrypted inputs.

\subsection{LUTs Evaluation}
\label{sec:primitives_hyppogriph}
Since the \gls{AES} evaluation involves computing an 8-bit \gls{S-box}, a straightforward solution would be to work with 8-bit messages. With such messages, the homomorphic \gls{S-box} evaluation would require only one bootstrapping per byte. However, processing messages with more bits significantly slow down the bootstrapping process. To address this issue, \cite{DBLP:conf/wahc/TramaCBS23} proposes a decomposition approach and demonstrates that the optimal representation of 8-bit inputs for their purpose is in $\Z_{16}$. Specifically, a message $m \in \{0, \cdots, 255\}$ is split into two 4-bit chunks (or \emph{nibbles}) $h$ and $l$ such that $m = 16h+l$. The encryption of $m$ is then represented as two ciphertexts encrypting $h$ and $l$ with the same key $\lweSecretKey$.

However, bootstrapping these decomposed inputs requires a method capable of handling multiple encrypted inputs. The authors explore several approaches for this, namely the chain-based method and the tree-based method \cite{TCHES:GuiBorAra21}. Their analysis concludes that the Tree-Based Method (\gls{TBM}) is the most suitable for their needs. They also relies on the \gls{MVB} to produce several outputs for the cost of one \gls{PBS}. We provide details about \gls{TBM} and \gls{MVB} in the following:

\paragraph{Multi-Value Bootstrapping from \cite{RSA:CarIzaMol19}.}
\label{primitive:mvb}
%
Multi-Value Bootstrapping (MVB) is a technique that enables the evaluation of $k$ distinct Look-Up Tables $(f_i)_{1 \le i \le k}$ on a single encrypted input, using only one $\BlindRotate$. This method is based on the factorization of the accumulator polynomials $\textsf{acc}_i(X)$ associated with each function $f_i$. Specifically, each accumulator polynomial is expressed as: 
$$
    \textsf{acc}_i(X) = \sum_{j=0}^{N-1} \alpha_{i,j} X^j, \quad \alpha_{i,j} \in \mathbb{Z}_q.
$$
The factorization then splits it into two parts: 
$$
    \textsf{acc}_i(X) = v_0(X) \cdot v_i(X) \mod (X^N + 1),
$$
where $v_0(X)$ is a common factor shared across all accumulators set as:
$$
    v_0(X) = \frac{1}{2} \cdot (1 + X + \dots + X^{N-1}),
$$
and $v_i(X)$ is a distinct factor specific to each function $f_i$:
$$
    v_i(X) = \alpha_{i, 0} + \alpha_{i, N-1} + (\alpha_{i, 1} - \alpha_{i, 0}) \cdot X + \dots + (\alpha_{i, N-1} - \alpha_{i, N-2}) \cdot X^{N-1}.
$$
This factorization is made possible thanks to the identity:
$$
(1 + X + \dots + X^{N-1}) \cdot (1-X) \equiv 2 \mod (X^N + 1).
$$
By leveraging this factorization and as illustrated on Figure~\ref{fig:mvb}, multiple LUTs can be evaluated on a single encrypted input by performing the following steps:
\begin{enumerate}
\item Computing a $\BlindRotate$ operation on an accumulator polynomial initialized with the value of $v_0$.
\item Then multiplying with \texttt{ClearMultTFHE} the obtained rotated polynomial by each $v_i(X)$ corresponding to the \gls{LUT} of $f_i$ to obtain the respective $\textsf{acc}_i(X)$.
\end{enumerate}
Finally, at the cost of a single $\BlindRotate$ and $k$ \texttt{clearMultTFHE} operations (on $\GLWE$), one can obtain the evaluation of $k$ different LUTs on one single encrypted input. Moreover, this specific choice of factorization allows for a very-low norm for the vectors $v_i$'s (which in practice are very sparse), and so a very-low noise expansion.

\begin{figure}
	\centering
	\begin{subfigure}[t]{\linewidth}
		\centering
		\mvbFigureA % <-- Replace with your TikZ or figure content
		\caption{Classic bootstrapping method to evaluate several LUTs on a single input}
		\label{fig:mvbA}
	\end{subfigure}
	
	\vspace{1.5em} % vertical space between figures
	
	\begin{subfigure}[t]{\textwidth}
		\centering
		\mvbFigureB % <-- Replace with your TikZ or figure content
		\caption{\gls{MVB}method}
		\label{fig:mvbB}
	\end{subfigure}
	
	\caption{Difference between the classical approach and the MVB. Pink arrows represent \texttt{clearMultTFHE} operations (on $\GLWE$).}
	\label{fig:mvb}
\end{figure}

This \gls{MVB}primitive thus allows significant speed-ups in the implementation of \cite{DBLP:conf/wahc/TramaCBS23}, in particular in the evaluation of the \gls{S-box} or in the multiplications in $\F_{256}$ that occur during the \MixColumns step. Indeed, since each byte is decomposed into two nibbles $h$ and $l$, the \gls{LUT} corresponding, for instance, to the \gls{S-box} must also be decomposed into two tables: one providing the most significant nibble and one providing the least significant nibble. That is to say: 
$$
\texttt{tab}_{\textsc{msn}}[i] = \left\lfloor \frac{\texttt{\gls{S-box}}[i]}{16} \right\rfloor \quad \text{and} \quad \texttt{tab}_{\textsc{lsn}}[i] = \texttt{\gls{S-box}}[i] \mod 16.
$$
Each of these tables must be evaluated on an 8-bit payload ciphertext. 


\paragraph{Tree-Based Method from \cite{DBLP:conf/wahc/TramaCBS23}.}
\label{prim:tbb}

Let $B, B', d \in \mathbb N^*$. The Tree-Based Method (\gls{TBM}) allows to evaluate a \gls{LUT} $f: \Z_{B^d} \mapsto \Z_ {B'}$ with a large input size $B^d$, by processing $d$ limbs of data in $\Z_B$. We consider input messages that are written as:
$$
    m = \sum_{i=0}^{d-1} m_i B^i, \quad \text{with } m_i \in \mathbb{Z}_B,
$$
and that are represented by $d$ ciphertexts $(\vec c_0, \vec c_1, \dots, \vec c_{d-1})$ corresponding to the $d$ message components $(m_0, m_1, \dots, m_{d-1})$. 
%
To evaluate $f$, we encode a \gls{LUT} for $f$ using $B^{d-1}$ accumulators, each represented by a polynomial $\textsf{acc}_i(X)$. These accumulators encode the functions:
%
    \begin{align*}
        f_i: \Z_B &\rightarrow \Z_{B'}\\
             x &\mapsto f(i + x \cdot B^{d-1})
    \end{align*}
%
Next, we apply a $\sf{BlindRotate}$ and a $\sf{SampleExtract}$ to each accumulator $\textsf{acc}_i(X)$, using $\vec c_{d-1}$ as the selector. This operation produces $B^{d-1}$ LWE ciphertexts, each encrypting $f (i + m_{d-1} \cdot B^{d-1})$ for $i \in \Z_{B^{d-1}}$.
%    
Finally, a $\sf{Keyswitch}$ operation from LWE to GLWE aggregates these ciphertexts into $B^{d-2}$ GLWE encryptions, representing the \gls{LUT} of $h$, defined as:
    \begin{align*}
        h &: (\Z_{B})^{d-1} \mapsto \Z_B' \\
          & (u_0, \dots, u_{d-1}) \mapsto f \circ g(u_0, \dots, u_{d-2}, m_{d-1})
    \end{align*}
using the bijection $g$, which reverses the decomposition:
    \begin{align*}
        g: (\Z_B)^d &\rightarrow \Z_{B^d} \\
           (u_0, \dots, u_{d-1}) &\mapsto \sum_{i=0}^{d-1} u_i \cdot B^i
    \end{align*} 

This process is repeated iteratively, using the next ciphertext at each step, until a single LWE ciphertext encrypting $f(m_0, \dots, m_{d-1})$ is obtained.  


\begin{figure}
    \centering
    \treePBSFigure
    \caption{Illustration of the tree-based method on messages  $m_1 = 1, m_2=2$ in the space  $\mathbb{Z}_4$. The corresponding ciphertexts are $\vec c_1 \in \LWE(m_1)$ and $\vec c_2 \in \LWE(m_2)$. We apply the addition in $\mathbb{Z}_4$ via programmable bootstrapping. Red arrows indicate bootstrappings. (Figure inspired by \cite{DBLP:conf/wahc/TramaCBS23}.)}
    \label{fig:my_label}
\end{figure}

In the implementation described in \cite{DBLP:conf/wahc/TramaCBS23}, this primitive is employed to evaluate an 8-bit \gls{LUT} by dividing it into two limbs of 4 bits each, which they determined to be optimal for their specific setting. 
%
To further enhance the performance of the \gls{TBM}, the blind rotations for the accumulators $acc_i(X)$ of \emph{the first layer of the tree} can be performed simultaneously using the \gls{MVB}technique (as discussed in~\cite{TCHES:GuiBorAra21}). 

Finally, the ``full-\gls{LUT}'' approach facilitates efficient computation of the \gls{S-box} through the Tree-Based Method, as opposed to directly evaluating the corresponding Boolean circuit. However, this approach also requires \gls{LUT}-based computation of XOR operations and other intermediary steps, which is notably slower when operating in $\Z_{16}$ compared to binary messages. Consequently, our new method \hippo{} proposed in this chapter strategically applies \gls{LUT} evaluation exclusively where it is most effective and yields the best performance, namely for the evaluation of the \gls{S-box}.




\section{Generalization of $p$-encodings to the Arithmetic Case}
\label{sec:generalization_p_encodings}

In Chapter \ref{chap:p_encodings} of this thesis, we introduced the notion of $p$-encodings and used it in Section \ref{sec:p_encodings_aes} to evaluate \gls{AES} homomorphically. In this method, data was encrypted bit per bit and only Boolean operations were performed. It leveraged the fact that, in the plaintext space $\Z_2$, the \texttt{SumTFHE} operation actually performs a XOR. Thus, the linear operations \MixColumns and \AddRoundKey could be efficiently performed with minimal cost, using only the homomorphic sum of \gls{TFHE}. To be able to evaluate $\SubBytes$ under Boolean representation, we used $p$-encodings to evaluate the circuits of \SubBytes with a minimal number of bootstrappings. While this has brought some improvements, evaluating the \gls{LUT} of \gls{AES} as a Boolean circuit is still suboptimal, and in this chapter we attempt at doing it using arithmetic representation.

To achieve that, we generalize $p$-encodings beyond the Boolean case by defining the $(o, p)$- encoding construction. Informally, instead of embedding the Boolean space in $\Z_p$, we embed any space $\Z_o$ in $\Z_p$ (with $o < p$). So, what was called $p$-encoding in Chapter \ref{chap:p_encodings} corresponds to a $(2, p)$-encoding in this one.
%
Definition \ref{def:hippogryph_encoding} formalizes this generalization.

\begin{definition}[$(o, p)$-encoding]
	Let $\Z_o$ be the message space. A \emph{$(o, p)$-encoding} is a function $\Encoding: \Z_o \mapsto 2^{\Z_p}$ that maps each element of $\Z_o$ to a subset of the discretized torus $\Z_p$. A $(o, p)$-encoding is \emph{valid} if and only if:
	\begin{equation}
		\begin{cases}
			\forall (i, j) \in \Z_o^2, i \ne j, \Encoding(i) \cap \Encoding(j) = \emptyset~\text{ and}\\
			\text{if $p$ is even:} ~\forall \:x \in \mathbb Z_p, \forall i \in \Z_o: x \in \Encoding(i) \iff \left [ x + \frac p 2 \right ]_p \in \Encoding([-i]_o)
		\end{cases}
	\end{equation}
	The latter property is a direct consequence of the negacyclicity problem, which we discussed extensively in Chapter \ref{chap:negacyclicity}.
	\label{def:hippogryph_encoding}
\end{definition}

In this work, we focus exclusively on cases where $p=2$ or $p$ is an odd prime. As a result, a lot of the subleties of negacyclicity can be overlooked. Furthermore, among the various types of $(o, p)$-encodings, one particular class proves especially useful for our purposes: the \textit{canonical} $(o, p)$-encoding.

\begin{definition}[canonical $(o, p)$-encoding]
	\label{def:canonical-encoding}
	A $(o, p)$-encoding $\Encoding$ is said \textit{canonical} if and only if it verifies: 
	\begin{align*}
		\Encoding: \Z_o &\rightarrow \Z_p\\
		x &\mapsto x
	\end{align*}
	(with $o < p$). Informally, we simply embed a smaller space into a larger one, without altering the order of the elements.
\end{definition}


In Chapter \ref{chap:p_encodings} the Boolean space is used (so $o=2$). The \SubBytes circuit is evaluated using $(2,11)$-encoding, while the rest is evaluated with a $(2, 2)$-encoding (\ie~the trivial encoding of \gls{TFHE} with plaintext space $\Z_2$). Consequently, an \textit{Encoding Switching} operation is required. This operation can be straightforwardly performed using a \gls{PBS}.

\begin{definition}[Encoding Switching]
	\label{def:encoding-switching}
	Let $\vec c$ be a ciphertext encrypting a message $m \in \Z_o$ under the $(o, p)$-encoding $\Encoding$. Its encoding can be switched to the $(o, p')$-encoding $\Encoding'$ by applying a \gls{PBS} on $\vec c$ evaluating the function:
	
	\begin{align*}
		\texttt{Cast}_{\Encoding \mapsto \Encoding'}: \Z_p &\rightarrow \Z_{p'}\\
		x &\mapsto x'
	\end{align*}
	where $x'$ is defined as $\forall i \in \Z_o, x \in \Encoding(i) \implies x' \in \Encoding'(i)$.
\end{definition}


