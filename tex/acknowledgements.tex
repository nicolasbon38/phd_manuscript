Je remercie particulièrement Caroline Fontaine et Pierre-Alain Fouque pour avoir accepté de rapporter ce manuscrit. Je remercie également Ilaria Chillotti, Adeline Roux-Langlois et Renaud Sirdey pour leur participation au jury de cette thèse.

J'ai beaucoup entendu que l'ingrédient le plus important pour une thèse réussie est la qualité de l'encadrement. Après ces trois dernières années, je peux confirmer que c'est absolument vrai. J'adresse donc un grand merci à mes trois encadrants : Sonia Belaïd, Matthieu Rivain et David Pointcheval. Merci Sonia, pour m'avoir fait confiance et accompagné depuis le premier jour de stage, quand je ne connaissais pas grand chose à la crypto, ni à la recherche. Merci pour tes encouragements constants et ton attention au quotidien. Merci Matthieu, pour ton inébranlable optimisme, ton inépuisable capacité à trouver des idées et tout simplement pour ta gentillesse. Enfin, merci David pour m'avoir accueilli à l'ENS. J'ai pu constater que ta réputation d'efficacité n'est pas usurpée, et je te remercie pour ta disponibilité pour mes questions malgré tout ce que tu gères. Merci à tous les trois. C'était très enrichissant et très agréable de travailler avec vous.

J'ai eu la chance de trouver un excellent environnement à CryptoExperts. Je remercie donc tous mes collègues anciens et actuels : Ryad Benadjila, Ghozlen Boukacem, Gaëtan Cassiers, Thibauld Feneuil, Louis Goubin, Viet-Sang Nguyen, Victor Normand, Pascal Paillier, Mélissa Rossi, Abdul Rahman Taleb, Muaad Tamtam, Ronan Thoraval et Auguste Warmé-Janville.

Faire une thèse Cifre signifie avoir un deuxième bureau. J'ai beaucoup apprécié faire partie de "l'open space" de l'ENS et de l'équipe qui y vit : donc merci à Léonard Assouline, Henry Bambury, Hugo Beguinet, Céline Chevalier, Cédric Geissert, Wissam Ghantous, Lenaïck Gouriou, Paul Hermouet, Laurent Holin, Antoine Houssais, Guirec Lebrun, Jules Maire, Brice Minaud, Ky Nguyen, Phong Nguyen, Paola de Perthuis, Eric Sageoli, Robert Schädlich, Erkan Tairi, Florian Tousnakhoff et Quoc-Huy Vu. Je remercie aussi Lise-Marie Bivard pour sa précieuse aide dans les différents méandres administratifs qu'il a fallu traverser.

Je tiens aussi à remercier mes co-auteurs : l'équipe du CEA pour nos réflexions sur l'AES: Aymen Boudguiga, Daphné Trama et Renaud Sirdey. Egalement, merci aux experts du symétrique de l'INRIA et de Versailles pour notre collaboration sur Transistor : Jules Baudrin, Christina Boura, Anne Canteaut, Gaëtan Leurent, Léo Perrin et Yann Rotella.

Merci à Christina Boura et Louis Goubin pour m'avoir donné l'opportunité d'enseigner, et aux étudiantes et aux étudiants de leur sympathie. Merci à Samuel Tap pour avoir pris le temps de m'expliquer les arcanes de TFHE et du paramétrage. Merci à Pierrick Méaux pour l'invitation à Luxembourg et pour les discussions sur le transchiffrement. Merci à Guirec Lebrun pour m'avoir fait découvrir MLS (et pour sa solidarité dans les différentes galères que cela a impliqué). Merci à Ryad Benadjila pour avoir été un très bon coach de C. Merci à l'équipe de développement de \texttt{tfhe-rs} pour ce fantastique outil dont j'ai usé et abusé.

L'obsessivité qui vient avec l'activité de recherche peut très vite dévorer les autres aspects de la vie, mais j'ai toujours pu compter sur mes proches pour me sortir la tête du guidon. Je remercie donc chaleureusement mes amis, des anciens du Nord-Isère aux Ensimagiens en passant par mes colocataires successifs, et ma famille, en particulier mes parents et ma petite soeur. Je termine avec une pensée pour Dalia, et la remercie pour tout ce qu'elle m'apporte en partageant ma vie. 





