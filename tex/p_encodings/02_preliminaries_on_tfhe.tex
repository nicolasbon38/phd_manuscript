\section{Preliminaries on Boolean Functions and Boolean Circuits}
\label{sec:p_encodings_preliminaries_boolean}


A Boolean function has the form $f: \B^\ell \longrightarrow \B$, with $\ell$ being called the \emph{arity} of the function. 

\begin{definition}
    
The Algebraic Normal Form (ANF) of a Boolean function $f: \{0,1\}^\ell \mapsto \{0,1\}$ is a polynomial expression in which each term corresponds to a specific input combination of $n$ variables. The ANF is defined as follows: \[f(x_1, x_2, \ldots, x_l) = a_0 \oplus a_1x_1 \oplus a_2x_2 \oplus \ldots \oplus a_{2^n-1}x_1x_2\ldots x_l\] \begin{align*}
\text{where: }a_0, a_1, a_2, \ldots, a_{2^\ell-1} & \in \{0,1\} \quad \text{are the Boolean coefficients and} \\
x_1, x_2, \ldots, x_\ell & \quad \text{are called the Boolean variables}
\end{align*}
\end{definition}

This result means that any Boolean function can be evaluated by the means of \texttt{AND} and \texttt{XOR} operations. In the following, we will focus on the implementation of Boolean circuits composed of these operations exclusively.


A Boolean function can be represented by its \emph{truth table}, which is simply a table gathering all the possible inputs and the corresponding result of the application by the function. It can also be represented with a Boolean formula. A third representation is the \emph{Boolean circuit}:

\begin{definition}
    A Boolean circuit associated to the Boolean function $f$ is a finite Directed Acyclic Graph whose edges are \emph{wires} and vertices are \emph{Boolean gates} representing Boolean operations. We consider \texttt{AND} gates and \texttt{XOR} gates, of fan-in 2 and fan-out 1. We also consider copy gates, of fan-in 1 and fan-out $>1$, that outputs several copies of its input. A circuit is further formally composed of input gates of fan-in 0 and fan-out 1, and output gates of fan-in 1 and fan-out 0.
    
    Evaluating a $\ell$-input $m$-output circuit consists in writing an input $\vec{x} \in \B^\ell$ in the input gates, processing the gates from input gates to output gates, then reading the outputs from the output gates.
	\label{def:boolean_circuit}
\end{definition}


This notion of Boolean circuit will be particularly useful in Section \ref{sec:p_encodings_graphs}.

