\section{State of the Art on Homomorphic Boolean Computations}


Let $f : \B^\ell \mapsto \B^{\ell'}$ be a Boolean function. There are two ways of handling its computations, and thus designing its homomorphic version

\begin{itemize}
	\item \textbf{As a multivariate table: } The programmable bootstrapping of TFHE can be seen as a homomorphic lookup table evaluation. So we can imagine evaluating the function all at once with a unique (and potentially large) bootstrapping..
	\item \textbf{As a Boolean circuit: } If we look at $f$ as a Boolean circuit (see Definition \ref{def:boolean_circuit}), then it is sufficient to design a homomorphic version of each Boolean gate. Then, it is quite easy to concatenate the blocks to evaluate any function we want, not only $f$
\end{itemize}


While both techniques seems straightforward, there are not a viable solution in practice. This section attemps to explain why, and presents the state of the art for the two strategy: evaluating $f$ as a LUT or as a Boolean circuit:


\paragraph{As a LUT:}


The algorithm of TFHE's programmable bootstrapping takes only one ciphertext in input, so one may think that it can be programmed only with univariate functions. However, there is an easy workaround allowing to evaluate multivariate functions.
To evaluate a $\ell$-bits LUT, you pick $p = 2^\ell$ for the plaintext modulus. Each input bit is encoded in the LSB of one value in $\Z_{2^\ell}$, and encrypted in a separate ciphertext. Now, to evaluate the function $f$, the $i$-th input is shifted into the $i$-th least significant bit by a multiplication by $2^i$. They are then all summed together to encode the value $b_{\ell-1}b_{\ell-2}\dots b_1b_0$ and the LUT can be processed by a bootstrapping on $\ell$ bits.

\TODO{Faire une figure illustrative}

This approach does not scale well when the number of inputs $\ell$ increases. Indeed, working in a larger plaintext space slows terribly the PBS algorithm. On the other hand, using a circuit representation ensures a constant time for the bootstrapping algorithm no matter the number of inputs. 


\paragraph{As a circuit:}

As we introduced in Section \ref{sec:p_encodings_preliminaries_boolean}, any Boolean function can be easily compiled into a Boolean circuit made of XOR and AND gates. 

If we naturally pick the plaintext modulus $p=2$, XOR operations are evaluable with the homomorphic sum of TFHE, which are computationally free (at the cost of a slight noise growth). However, evaluating the non-linear gate AND requires bootstrapping. As we explained in the previous paragraph, bootstrapping is a univariate operation but it is possible to turn it into a bivariate one by using two bits of plaintext instead of one (so $p=4$). But this is not the end of it ! Because 4 is even, the negacyclicity problem applies and we need a third bit to work as a padding bit, so working with $p=8$.

The problem now is that XOR can no longer be evaluated with a simple sum, because carries would propagate into the MSB and ruin the correctness. So one would need \textit{one bootstrapping per Boolean gate} to evaluate the circuit while keeping the MSB clean. This is the solution implemented in \texttt{tfhe-rs} library \cite{tfhe-rs}\footnote{more precisely, in the \texttt{boolean} crate.}. 


\cite{AC:CLOT21} proposes a different approach: by leveraging a newer version of the TFHE scheme supporting multiplications of $\LWE$ ciphertexts, Boolean circuits are evaluated with homomorphic sums for \texttt{XOR} gates and this new multiplication operation for \texttt{AND} gates. While this approach is clearly a progress from the vanilla framework, we note that a few bootstrappings are still required to control the noise growth and that this new operation of $\LWE$ multiplications remains costly both in terms of performances and in terms of noise. Thus, we choose to stick to the first version of the TFHE scheme to keep the framework lighter and we tackle the performance issues of \cite{JC:CGGI20} with a different approach than the one of \cite{AC:CLOT21}.

\bigskip

We have just seen that both approaches do not scale well for different reasons: LUT because of bootstrapping complexity and circuit because of the number of bootstrapping occurences. Our approach attempts to gain the best of both worlds: we construct a constant-time bootstrapping for any number of inputs (solving the LUT problem), while using only one bootstrapping to evaluate the entire function (solving the circuit problem).

