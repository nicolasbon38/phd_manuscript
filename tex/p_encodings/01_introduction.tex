% !TeX_ROOT=../../thesis.tex


TFHE being the most efficient scheme to handle data of small precision, it is a natural choice when it comes to evaluate homomorphically Boolean circuits. However, performances of the existing frameworks are still limited. 

The most straightforward method, already introduced in the original TFHE paper \cite{JC:CGGI20} under the name \emph{gate bootstrapping}, consists in running one bootstrapping for each bivariate Boolean gate of the circuit. As a consequence, the conversion of the original Boolean circuit in a homomorphic circuit handling encrypted bits is straightforward, moreover the noise growth is contained thanks to the systematic use of bootstrapping. However, this approach is very expensive due to the high numbers of bootstrappings and makes it quite suboptimal for large circuits.


In this chapter, we introduce a new framework to homomorphically evaluate Boolean functions on encrypted data efficiently, i.e. by reducing the amount of necessary bootstrappings. Our approach introduces an \emph{intermediate homomorphic layer} which encodes the bits (elements of $\Z_2$) on a small ring $\Z_p$ before encrypting them. This allows us to evaluate Boolean functions with one cheap homomorphic sum followed by one bootstrapping. After formalizing the underlying concept of $p$-\emph{encoding} and explaining our evaluation strategy, we investigate the issue of finding valid sets of encodings for a Boolean function. We characterize this problem and provide an exact constructive algorithm to solve it. We further provide a sieving heuristic finding solutions more efficiently but at the cost of loosing optimally. Since our method is only efficient for Boolean functions with limited number of inputs, we also propose a heuristic to decompose any Boolean circuit into Boolean functions which are efficiently evaluable using our approach. Finally, we apply our technique to various cryptographic primitives, namely the SIMON block cipher, the Trivium stream cipher, the Keccak permutation, the Ascon S-box and the AES S-box. Compared to previous works implementing the same primitives (for SIMON, Trivium and AES) our implementations achieve significant speedups.

After some technical preliminaries on Boolean circuits (Section \ref{sec:p_encodings_preliminaries_boolean}), we introduce a new concept of \emph{intermediate homomorphic layer} and explain how bits are encoded  in Section \ref{sec:p_encodings_homorphic_layer} and the algorithms to construct it in Sections \ref{sec:p_encodings_search} and \ref{sec:p_encodings_graphs}. Finally, we describe some implementation works in Section \ref{sec:p_encodings_TFHE_adaptation} and our experimental results in Section \ref{sec:p_encodings_implementations}.










