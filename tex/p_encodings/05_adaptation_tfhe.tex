\section{Implementation issues: Adaptation of the parameters and of the \texttt{tfhe-rs} Library}
\label{sec:p_encodings_TFHE_adaptation}


From a high level point of view, our technique can be seen as adding an additional layer of abstraction on top of TFHE. It remains to explain how to select the parameters for the TFHE scheme. Moreover, to implemented our framework, we had to fork the \texttt{tfhe-rs} library~\cite{tfhe-rs}. The following section covers these two issues.



\subsection{Crafting of Parameters}
\label{sec:parametrization}

In practical setting, we need a set of parameters for TFHE. Finding an optimal set of parameters for a given application is a hard problem, that we study in depth in Chapter \ref{chap:parameters}. This work being anterior to the one of Chapter \ref{chap:parameters}, we used the framework of \cite{AC:CLOT21} and implemented in \cite{concrete-optimizer}.


In this paper, the authors elaborate a strategy where they define an \textit{atomic pattern} of FHE operators, that is to say a subgraph of FHE operators in which the noise of the output is independent from the one in the inputs. Then, they develop an optimization framework to derive the best set of parameters for a given atomic pattern.
%
In particular, the first atomic pattern they study, that they denote by $\mathcal{A}^{(CJP21)}$, is a subgraph composed of a linear combination of ciphertexts with clear constants, then a \texttt{Keyswitch} and then a \texttt{BlindRotate} followed by a \texttt{SampleExtract} (\texttt{ModulusSwitch} is seen as a part of \texttt{BlindRotate}). To dimension the parameters of TFHE to evaluate such an atomic pattern, their framework takes as input the 2-norm of the vector of constants of the linear combination (denoted by $\nu$) and a noise bound $t$ on the standard deviation of the distribution of error in a ciphertext that ensures a correct decryption with a good probability $(1-\epsilon$). We elaborate further on how this bound is constructed below in this section.
%
If we look closely, the evaluation of a gadget we introduced in Definition \ref{def:gadget} can be seen as a $\mathcal{A}^{(CJP21)}$ with a few differences. Thus, we slightly modified the tool \texttt{concrete-optimizer} \cite{concrete-optimizer}, that allows to generate parameters for different types of atomic patterns, to support our gadget as a new atomic pattern. Let us dive into the differences between a gadget and a $\mathcal{A}^{(CJP21)}$:
%
\paragraph{Support of odd values for $p$:} Using an odd value for $p$ changes the bootstrapping procedure, and in particular the definition of the accumulator for the \texttt{BlindRotate} (as explained in Chapter \ref{chap:negacyclicity}). With our construction, the windows in the polynomial are half the size of the ones for an even $p$, which impacts the noise bound $t$. 
As this bound depends of the failure probability $\alpha$ that the user is ready to tolerate, we shall denote it $t_\alpha$ hereafter, which satisfies: $t_\alpha = \frac{\Delta}{2z^*(1-\sqrt[N]{1-\alpha})}$
where $z^*$ is the \emph{standard score} and $\Delta$ is the scaling factor (see~\cite{AC:CLOT21} for more explanations). The impact of our adaptation on this formula is solely with respect to the scaling factor. In the context of an $\mathcal{A}^{(CJP21)}$, we have $\Delta = \frac{q}{2^\pi p}$ with $\pi$ the number of MSB for padding. As explained in Chapter \ref{chap:negacyclicity}, we do not need any padding mechanism anymore, so the $2^\pi$ vanishes. However, the length of a window is divided by $2$, and $p$ does not divide $q$ anymore so we need to add a rounding. We finally get $\Delta = \rounding{\frac{q}{2p}}$.

%
\paragraph{Link between input encodings and $\nu$:} In a scenario where only one gadget has to be evaluated, its inputs are freshly encrypted ciphertexts. Then, there is no need to perform any encoding switching before evaluating the gadget, and so we are in the context of a $\mathcal{A}^{(CJP21)}$ with $\nu = 1$. However, if we are in a context of a \textit{graph} of gadgets like in Section \ref{sec:p_encodings_graphs}, the output of a gadget can be used as input of subsequent gadgets under different encodings. In this case, some encoding switchings are necessary. If these encoding switching are made using a mutiplication by a constant (Property \ref{prop:mult_constant}), we are still in the context of a $\mathcal{A}^{CJP21}$ but with $\nu \ne 1$. 
To formalize that, we first recall that Algorithm \ref{alg:add_element} produces gadgets of the form $\Gamma = \left (\vec{\Encoding_{in}}, \Encoding_{out}, p_{in}, p_{out}, f \right )$, with $\Encoding_{in}^{(i)} = \EncDefOne{d_i}$. Thus, if we fix that all gadget output ciphertexts are encoded under $\Encoding_{out} = \EncDefOne{1}$, then the encoding switchings needed before an evaluation of $\Gamma$ corresponds to a linear combination of the inputs with the vector $\vec d = (d_i \mid i \in [1, \ell])$, so we fall back on a $\mathcal{A}^{(CJP21)}$ with $\nu = \norm{\vec d}$.


We implemented these changes in \texttt{concrete-optimizer} and uses it to generate sets of parameters for our implementations detailed in Section \ref{sec:p_encodings_implementations}.

\subsection{Concrete Implementations of $p$-Encodings and Homomorphic Functions in \texttt{tfhe-rs}}
\label{sec:library}

To implement our framework, we relied on the $\texttt{tfhe-rs}$ library~\cite{tfhe-rs}. Here is a list of the major changes we applied to the code:

\paragraph{Addition of the notion of $p$-encoding: } An encoding $\Encoding$ is simply implemented with a structure \texttt{Encoding} storing two \texttt{HashSets} and the modulus $p$. The \texttt{HashSets} represent both sets $\Encoding(0)$ and $\Encoding(1)$. When creating an \texttt{Encoding}, the code checks whether the two underlying sets are disjoint or not. Moreover, the operation of encryption and decryption are modified as well. The signatures change from:\[\texttt{encrypt(Boolean, ClientKey) -> Ciphertext}\] to: \[\texttt{encrypt(Boolean, ClientKey, Encoding) -> Ciphertext}\] (same for \texttt{decrypt}). The functions also perform the mapping $\B \mapsto \Z_p$ before encryption and the other way around after decryption.


\paragraph{Support of odd moduli: } The native \texttt{tfhe-rs} only support power-of-two-moduli $p$. We extended the library to handle odd values for $p$. This required modifying the encryption and decryption algorithm, and to compute the sets of parameters with the method of Section \ref{sec:parametrization}.


\paragraph{Definition of the new structure $\texttt{Gadget}$: } According to the evaluation strategy we introduced in Section \ref{sec:p_encodings_new_strategy}, we wrote a new structure $\texttt{Gadget}$, associated to a Boolean function $f: \B^\ell \mapsto \B$, carrying:
\begin{itemize}
    \item A list of the \texttt{Encoding} objects for the inputs: $\Encoding_{in} = (\Encoding_1, \dots, \Encoding_l)$, with the input modulus $p_{in}$ they encoded on.
    \item The output \texttt{Encoding} object $\Encoding_{out}$, with the output modulus $p_{out}$ it is encoded on.
    \item The clear function $f$.
\end{itemize}
When such a structure is constructed, it self-checks whether $f(\Encoding_{in})$ is valid. Then, when provided $\ell$ $\texttt{Ciphertexts}$ objects encoded under their respective $p$-encoding, it executes the homomorphic sum and the PBS and outputs the results encoded under $\Encoding_{out}$. Some utilitary functions performing encoding-switching are also available, allowing the chaining of several $\texttt{Gadget}$.


\paragraph{Implementation of the accumulator: } The procedure of bootstrapping of \texttt{tfhe-rs} is slightly modified to support the new version of the accumulator we introduced in Chapter \ref{sec:negacyclicity}.

\paragraph{Parsing of graphs: } We implemented a Python script that produces graphs to represent more complex functions that requires several PBS, as described in Section \ref{sec:p_encodings_graphs}. These graphs are stored with a comprehensive file format and our Rust implementation has a module of parsing allowing to load these graphs and automatically generate the corresponding graph of \texttt{Gadget}.



