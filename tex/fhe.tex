\chapter{Introduction: Fully Homomorphic Encryption}



\section{Why ?}





\section{Historical Background}






\section{The bootstrapping : breakthrough construction}





\section{Security Properties}



It is common ni cryptography to modelize the security of a scheme by a \textit{security game}. In an idealized world, a challenger running the scheme is opposed to a polynomial-time attacker. This attacker has access to \textit{oracles} that they can query. For encryption scheme, one of these game is the \textit{indisguishability game}: the attacker pick two messages and send them to the challenger. The challenger flips a coin and encrypts one of the messages at random. They send this ciphertext (named the \textit{challenge ciphertext}) to the attacker, who have to guess which message has been encrypted. If the attacker can do better that random guessing (so guess right with probability larger than 0.5), we say that they have a \textit{non-negligible advantage}, indicating a vulnerability against the type of attack.

Usually, the ``graal'' for encryption schemes is \textsf{CCA2} security (or \textit{adaptative chosen-ciphertext security}). In the corresponding game, the attacker has access to an encryption and a decryption oracle, that they can query both before and after having received the challenge ciphertext, with the exception that they are forbid to query the decryption oracle directly with the challenge ciphertext.

By design, homomorphic schemes cannot achieve such security property because they are intrinsically \textit{malleable}. So a trivial attack would be to homomorphically add a zero value to the challenge ciphertext and query the decryption oracle with the resulting ciphertext. The oracle would accept the decryption and output the message, breaking the security of the scheme.

Actually, things are even worse: if the scheme is bootstrappable, then even \textsf{CCA1} security is unachievable. This corresponds to a \textit{non-adaptative chosen-ciphertext security}), the difference with \textsf{CCA2} being that the attacker is no longer allowed to query the decryption oracle after having seen the challenge ciphertext (so it modelizes a weaker security property). Yet, a simple attack consists in querying the decryption oracle with the bootstrapping key. Recall that in Gentry's blueprint, the bootstrapping key is an encryption of the secret key. Thus, the attacker may recover the secret key and decrypt any ciphertext sent by the challenger.


Because these advanced security properties are so hard to achieve, the \textit{de facto} standard for FHE schemes has become \textsf{CPA} security (where the attacker has only access to an encryption oracle and no decryption oracle). This is the level of security targeted by ``mainstream'' libraries such as ...

However, as underlined in \cite{bibid}, such security level is not sufficient for some use-cases of FHE: ....


Attempts have been made to extend the security capabilities of Fully Homomorphic Encyrption further. The first 


