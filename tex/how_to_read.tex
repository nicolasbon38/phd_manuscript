%!TeX_ROOT=../thesis.tex

\paragraph{How to read this thesis ?}
\TODO{Aussi introduire l'intro en Français}


Chapter \ref{chap:fhe} provides an introduction on the topic of Fully Homomorphic Encryption, including some historical considerations and a presents a state of the art.

Chapter \ref{chap:tfhe} introduces TFHE, the cryptosystem on which this thesis focusses. It presents its inner workings, as well as the challenges of implementations. The \textit{Programmable Bootstrapping}, which is at core of TFHE's homomorphic capabilities, is presented in detail.

Chapter \ref{chap:negacyclicity_and_oddness} presents the negacyclicity problem, one of the main issue constraining the use of TFHE in practice. We give a formal presentation of the problem, as well as an overview of the state of the art of the solutions present in the literature. We introduce a new method called the \textit{odd plainext modulus}, which solves the negacycliity problem while enabling new features of TFHE. This construction is the building block upon which the contribuitions of the rest of this thesis are built.

Chapter \ref{chap:p_encodings} presents a method to accelerate the evaluation of arbitrary Boolean functions in TFHE. It relies on so-called $p$-encodings embedding the bits into a prime field to better leverage TFHE's homomorphic operations. We apply our method to some cryptographic primitives and demonstrates a significant gain of performances with respect to the state of the art. 
This work has led to a publication in 
\TODO{Ici copier le bout de biblio}.


Chapter \ref{chap:hippogryph} extends this result further by generalizing the encoding method beyond the Boolean case to the arithmetic one. In this work, we attempted at constructing a fast homomorphic implementation of the Advanced Encryption Standard (AES) cipher. To achieve that, we leverage both Boolean and arithmetic representations and develop a generic framework to efficiently switch from one to another. We successfully exhibit the fastest implementation of AES of the literature.
This work has led to a publication in 
\TODO{Ici copier le bout de biblio}.


Chapter \ref{chap:transistor} presents \texttt{Transistor}, a TFHE-friendly steam cipher. We give its specification and explain the rationale behind the design choices. The largest part of this chapter is dedicated to analyze the good performances of \texttt{Transistor} in the homomorphic domain. 
This work has led to a publication in 
\TODO{Ici copier le bout de biblio}.
The aforementioned published version includes much more material, including an in-depth security analysis of the cipher.


Chapter \ref{chap:larger_lut} presents a method to accelerate the homomoprhic evaluation of large look-up tables, extending the capabilities of the original Programmable Bootstrapping ot the TFHE cryptosystem.
\TODO{Détailler quand le chapitre sera écrit}


Finally, Chapter \ref{chap:parameters} goes beyond the scope of the ``odd plaintext modulus'' framework and introduces \toolName, a tool designed to help FHE practitioners to create parameter sets ensuring the three central properties: security, correctness of the computations and efficiency.
\TODO{Détailler quand ce sera écrit}






 