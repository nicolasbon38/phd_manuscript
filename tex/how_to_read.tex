%!TeX_ROOT=../thesis.tex

\chapter*{How to read this thesis ?}
\addstarredchapter{How to read this thesis ?}

The thesis begins by a brief section written in French, that introduces the topic and provides a summary of the contributions of this thesis.

Chapter \ref{chap:fhe} is an introduction presenting \gls{FHE}, including some historical background and a review of the state of the art. Then, Chapter \ref{chap:spec_tfhe} introduces the \gls{TFHE} cryptosystem in detail, presenting all its internal components and their functioning. In particular, its bootstrapping operation, which lies at the heart of its homomorphic capabilities, is described in depth.

In Chapter \ref{chap:negacyclicity}, we address one of \gls{TFHE}’s fundamental issues: the negacyclicity problem. This is one of the main hurdles when using \gls{TFHE} in practice, as it limits the performance of homomorphic operations and greatly complicates the design of homomorphic programs. We provide a formal presentation of the problem as well as an overview of the existing solutions found in the literature. We then introduce a new approach using a plaintext space of odd order, which resolves the negacyclicity issue while also enabling new functionalities for \gls{TFHE}. This construction forms the foundation on which the rest of this thesis builds.

Chapter \ref{chap:p_encodings} presents a method to accelerate the evaluation of arbitrary Boolean functions in \gls{TFHE}. The core technique of \gls{TFHE}’s original paper performs one bootstrapping per logic gate in the function’s circuit. The problem is that this approach does not scale well when working with more complex functions or more inputs. To overcome this, we developed a new type of encoding, called $p$-encodings, which embed bits into a larger space. This allows multiple bits to be compressed into the same ciphertext by summing them, enabling evaluation of the entire function through a single bootstrapping. We develop algorithms to find suitable $p$-encodings for a given function. If the circuit is still too large, we also present an algorithm to decompose it into sub-blocks that can be processed using our method. To test this construction, we apply it to several cryptographic primitives and demonstrate significant performance improvements compared to the state of the art.

This work resulted in the publication:

\begin{center}
\fullcite{TCHES:BonPoiRiv24}
\end{center}

Chapter \ref{chap:hyppogriph} aims to improve the homomorphic implementation of the \gls{AES} standard from the previous chapter. To do so, we exploit both Boolean and arithmetic representations and develop a general framework for efficiently switching between the two. This requires generalizing the encoding method from the previous chapter beyond the Boolean case to the arithmetic case, and adapting advanced homomorphic operators from the literature to this encoding strategy. This leads to the fastest \gls{AES} implementation in the literature.

This work resulted in the publication:

\begin{center}
\fullcite{hippogryph}
\end{center}

The main target use case of Chapter \ref{chap:hyppogriph} is transciphering, a cryptographic technique that solves the problem of ciphertext expansion. When data is encrypted homomorphically, it occupies much more memory space and thus consumes more bandwidth when sent to a server. Transciphering addresses this issue: instead, the client encrypts the data using a conventional symmetric cipher, and the server homomorphically decrypts the data to bring it into the homomorphic domain. Experimental results show that using a standard cipher like \gls{AES} is not very efficient; instead, one would prefer a cipher specifically designed to be efficiently evaluable under homomorphic encryption. This is exactly what we construct in Chapter \ref{chap:transistor}: we present our contribution to the design of \texttt{Transistor}, a stream cipher that is highly efficient under \gls{TFHE}. We provide its specification and explain the rationale behind its design choices, including the use of an odd modulus. Most of the chapter is dedicated to analyzing the strong performance of \texttt{Transistor} in the homomorphic domain.

This work was published in:

\begin{center}
\fullcite{transistor}
\end{center}

The published version above includes significantly more content, including an in-depth security analysis of the scheme.

One of \gls{TFHE}’s main limitations is that programmable bootstrapping becomes very slow as the size of the message increases. As a result, evaluating Look-Up Tables (LUTs) larger than 8 bits is impractical. In Chapter \ref{chap:larger_lut}, we tackle this issue by extending \gls{TFHE}’s bootstrapping capabilities beyond 8 bits through a method that accelerates homomorphic evaluation of large LUTs. Once again leveraging our encoding technique, we design a \gls{LUT} decomposition algorithm that enables bootstrapping to be applied on smaller messages.


Finally, Chapter \ref{chap:parameters} goes beyond the scope of our encoding method in odd spaces and introduces \toolName, a tool designed to help homomorphic program designers choose parameter sets that ensure the three central properties: security, correctness of computation, and efficiency. The strength of \toolName lies in its flexibility, allowing easy extension to new homomorphic operators. Furthermore, its optimization algorithm estimates runtime using a cost model that depends on the machine executing the program, allowing parameter sets to be tailored to various usage contexts.
