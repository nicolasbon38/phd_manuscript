%!TEX root = ../thesis.tex

\chapter{Overcoming the negacyclicity problem with odd plaintext modulus}
\label{chap:negacyclicity}

In Section \ref{sec:pbs}, we presented the algorithm of the PBS. But we hid something under the rug: the actual composition of the accumulator polynomial $\acc$. This definition depends on the countermeasure picked against the negacyclicity problem (whom we had a glimpse on  in Section \ref{sec:overview_blind_rotation}. 


In this chapter, we formalize the negacyclicity problem. We then present the classical countermeasure (the strategy of the padding bit), and our technique of odd moduli, which is a building block for the rest of the works presented in this thesis. We finish by going through several others works that propose different countermeasures.




\section{Basics on negacyclicity}

\paragraph{What is negacyclicity ?}

Let $v(X)$ be a polynomial of the ring $\glweRing$, denoted by $v(X) = \sum_{k=0}^{N-1} v_k X^k$. Recall how a multiplication by $X$  in this ring ``rotates'' the coefficients of the polynomial: \[X \cdot v(X) = - v_{N - 1} + v_0 \cdot X \dots + v_{N - 2} X^{N - 1}~.\]

In TFHE's blind rotation, the polynomial multiplication in the blind rotation is actually done by $X^{-\Tilde{\mu}}$, which lives in $\{0, \dots, 2N - 1\}$. This leads to two problems:

\begin{itemize}
	\item A coefficient $v_j$ can be brought in first place by two differents rotations: the one induced by the polynomial multiplication by $X^{\modulo{-j}{2N}}$ and the one by $X^{[-j + N]_{2N}}$.
	\item Each time a coefficient goes last to first, it gets negated (because $X^N = -1$ in the ring). So actually, the multiplication by $X^{[-j]_{2N}}$ yields correctly $v_j$, but the one by $X^{[-j + N]_{2N}}$ yields $-v_j$.
\end{itemize}


As the actual value of $\tilde{\mu}$ is encrypted, this is not possible to predict beforehand whether this undesirable minus sign will appear or not. So, the accumulator polynomial has to be constructed in such a way that this problem is neutralized.


\paragraph{The ideal case}

Some functions interact naturally very well with the blind rotation algorithm: these are called the \textit{negacyclic functions} and they are presented in Definition \ref{def:negacyclic_function}.


\begin{definition} (Negacyclic Function)
	Let $p$ and $p'$ be two positive integers, with $p$ even. A function $f : \Z_p \mapsto \Z_{p'}$ is negacyclic if and only if it satisfies the following property:
	
	\[
		\forall x \in \Z_p, f\left (\modulo{x + \frac p 2}{p} \right ) = \modulo{-f(x)}{p'}
	\]
	
	\label{def:negacyclic_function}
\end{definition}


With such functions, the accumulator is quite simple to build: intuitively we fill it with the values of the upper half of the torus. By the negacyclic property, the minus signs get cancelled when the rotation occurs.

\begin{definition} (Negacyclic Accumulator)
Let $p$ and $p'$ be two positive integers, with $p$ even. Let $f:\Z_p \mapsto \Z_{p'}$ be a negacyclic function. Let $N$ be a power of two. Then, the accumulator $\acc$, defined by:
	
	\[
		\acc^{\textsf{negacyclic}} = X^{\frac{-2N}{2p}} \cdot \sum_{j=0}^{2N / p - 1} X^j \cdot \sum_{i=0}^{p/2 - 1} f(i) X^{i \frac{2N}{p}} \mod (X^N + 1)
	\]
	
	is a valid accumulator for the blind rotation.
\end{definition}

\TODO{Write a small proof}

This construction is merely theoretical. Indeed, in practical setting, restricting the functions to be evaluated by PBS only to negacyclic function greatly limits the capability of the scheme. Thankfully, some countermeasures against the negacyclicity problem exist, allowing to evaluate \textit{any} function in the PBS.


\section{The classical countermeasure: the bit of padding}

The first countermeasure, that already appeared in the original TFHE paper, is called the \textit{bit of padding}. It works in even plaintext space.

The idea can be summed up by "if we are always in the first half of the torus, the minis signs cannot hurt us''. This is equivalent to forcing the MSB of the message to be zero. The adventage of this method is that it relaxes the negacyclicity constraint and makes the bootstrapping able to evaluate any function (not only the negacyclic one). The drawback is that any linear function can break it by propagating a carry in the MSB. So, when constructing homomorphic circuits one needs to keep track of the maximal value that each homomorphic operation can yield and make sure that it will not propagate a carry into the MSB. This often leads to extra bootstrapping to clear the MSB. In Chapter ]\ref{chap:p_encodings}, we will show how this method makes the evaluation of Bollean circuit very inefficient, and what we can do to improve that.


In the following, we give the definition of the accumulator $\acc$ in those cases:

\begin{definition}(Bit-of-Padding accumulator)
	TOWRITE
	\[
		\acc^{\textsf{bit-of-padding}} = X^{\frac{-N}{2p}} \cdot \sum_{j=0}^{N / p - 1} X^j \cdot \sum_{i=0}^{p - 1} f(i) X^{i \frac{N}{p}} \mod (X^N + 1)
	\]
\end{definition}



\TODO{FAIRE UNE JIOLIE FIGURE}


\TODO{

		- works in a even plaintext space
		- expliquer comment ça marche
		-  donner la tête de l'accumulateur
		- mais contraintes sur les circuits homomorphes que l'on peut évaluer

}


\section{A new solution: the odd plaintext modulus}

- copy paste section 6.1 of BPR23



\paragraph{A tour of other solutions}

- Schild
- Le WoPBS (mais peut-être le présenter plus longuement dans le papier LUT)






