% !TeX_ROOT=../thesis.tex

\newcommand{\TODO}[1]{\todo[inline, color=orange!30]{\textbf{TODO:} #1}}
\newcommand{\Question}[1]{\todo[inline, color=red!30]{\textbf{Question:} #1}}



% Classical Spaces
\newcommand{\B}{\mathbb{B}}
\newcommand{\Z}{\mathbb{Z}}
\newcommand{\T}{\mathbb{T}}
\newcommand{\R}{\mathbb{R}}

% Distributions
\newcommand{\unif}[1]{\mathcal U \left ( #1 \right )}


% Problems / Ciphertexts
\newcommand{\LWE}{\textsf{LWE}}
\newcommand{\TrivialLWE}{\textsf{TrivialLWE}}
\newcommand{\RLWE}{\textsf{RLWE}}
\newcommand{\GLWE}{\textsf{GLWE}}
\newcommand{\TrivialGLWE}{\textsf{TrivialGLWE}}
\newcommand{\GGSW}{\textsf{GGSW}}

% Classical Mathematical Operations
\newcommand{\rounding}[1]{\left \lfloor #1 \right \rceil}
\newcommand{\modulo}[2]{\left [ #1 \right ]_{#2}}
\newcommand{\drawfrom}{\overset{\$}{\leftarrow}}



% TFHE-specific notations
\newcommand{\lweSigma}{\sigma_{\LWE}}
\newcommand{\glweSigma}{\sigma_{\GLWE}}
\newcommand{\baseDecomp}{B}
\newcommand{\levelDecomp}{\ell}


\newcommand{\lweSecretKey}{\vec s}
\newcommand{\glweSecretKey}{\vec S}

\newcommand{\BSK}{\textsf{BSK}} 
\newcommand{\KSK}{\textsf{KSK}} 
% TODO : trouver une otation pl^us jolie pour BSK


%TFHE homomorphic operations

\newcommand{\sumTFHE}[2]{\texttt{SumTFHE}(#1, #2)}
\newcommand{\sumTFHETernary}[3]{\texttt{SumTFHE}(#1, #2, #3)}
\newcommand{\sumTFHEQuad}[4]{\texttt{SumTFHE}(#1, #2, #3, #4)}
\newcommand{\clearMultTFHE}[2]{\texttt{ClearMultTFHE}(#1, #2)}

\newcommand{\squarewithdot}{\tikz[baseline=-0.5ex]\draw (0,0) rectangle (0.3,0.3) (0.15,0.15) circle (0.05);}

% Bootstrapping operations
\newcommand{\ModSwitch}{\texttt{ModSwitch}}
\newcommand{\BlindRotate}{\texttt{BlindRotate}}
\newcommand{\SampleExtract}{\texttt{SampleExtract}}
\newcommand{\CMUX}{\texttt{CMUX}}

% other operations
\newcommand{\decomp}[3]{\textsf{dec}^{(#1, #2)}(#3)}

% TFHE-related spaces\\
\newcommand{\plaintextTorus}{\mathbb{T}_p}
\newcommand{\plaintextRing}{\mathbb{Z}_p}
\newcommand{\plaintextTorusPoly}{\mathbb{T}_p[X] / (X^N + 1)}
\newcommand{\plaintextRingPoly}{\mathbb{Z}_p[X] / (X^N + 1)}
\newcommand{\lweTorus}{\mathbb{T}_q}
\newcommand{\lweRing}{\mathbb{Z}_q}
\newcommand{\glweTorus}{\mathbb{T}_q[X] / (X^N + 1)}
\newcommand{\glweRing}{\mathbb{Z}_q[X] / (X^N + 1)}



% environments
\theoremstyle{definition}
\newtheorem{definition}{Definition}[section]