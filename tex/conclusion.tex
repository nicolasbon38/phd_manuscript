\chapter*{Conclusion}
\addstarredchapter{Conclusion}

Fully Homomorphic Encryption is believed to be on the verge of practical usability. However, several challenges have still to be overcome before it can be integrated into everyday applications. The purpose of this thesis was to adress some concrete obstacles that hinder the practical deployment of \gls{FHE}. In this concluding section, we list the main challenges and relate them to the contributions presented throughout this manuscript.


\paragraph{On Efficiency.}

\gls{FHE} still incurs an significant computational overhead compared to traditional, unencrypted computation. While the \gls{FHE} schemes themselves have been seen much improvement in the last years, an orthogonal direction is to design new algorithms tailored for specific use-cases. In Chapter \ref{chap:p_encodings}, we developed a framework to accelerate the evaluation of Boolean functions. At the opposite end of the spectrum, Chapter \ref{chap:larger_lut} focus on accelerating the evaluation of \gls{LUT} in larger plaintext spaces than those originally supported by \gls{TFHE}. Chapter \ref{chap:hyppogriph} further demonstrated how Boolean and arithmetic representations offer complementary advantages, and how efficient conversion mechanisms between both can enhance performance within homomorphic circuits. 

Another way of improving performances lies in selecting appropriate parameters for the scheme. We propose such a procedure of parameter selection, which takes into account the computational circuit to evaluate as well as the environment of execution.


\paragraph{On Data Expansion and Transciphering.}


Data expansion is another well-known bottleneck in \gls{FHE}. The literature has long proposed transciphering as a promising solution, however this technique necessitates homomorphic evaluation of the decryption function of a symmetric cipher. In Chapter \ref{chap:hyppogriph}, we have shown how far we could push to evaluate efficiently the \gls{AES} cipher using \gls{TFHE}. However, it is clear that standard ciphers such as \gls{AES} cannot compete against schemes specifically designed with the use-case of transciphering in mind. In Chapter \ref{chap:transistor}, we present such a cipher and show how its design combines the properties required to ensure both cryptographic security and homomorphic efficiency. 




\paragraph{On Compilation and Development of Homomorphic Applications.}
%

Developing homomorphic applications remains a tedious task, requiring deep understanding of both cryptography and the internal workings of specific \gls{FHE} schemes. It is clear that an adoption of \gls{FHE} at scale will require an automated compilation toolchain that will abstract away cryptography complexities from non-expert programmers.

In Chapter \ref{chap:p_encodings}, we proposed algorithm that automatically compiles Boolean functions into optimized sequences of homomorphic operations. We did a similar thing in the case of large LUTs in Chapter \ref{chap:larger_lut}, where those LUTs are compiled into a more tractable algorithm, composed of smaller ones. Finally, our parameter selection framework presented in \ref{chap:parameters} further contributes to this effort.


\bigskip

The road ahead is likely still long before \gls{FHE} becomes a ubiquitous technology. However, the pace of scientific progress in this field is accelerating rapidly, making it reasonable to believe (and hope) that such a technological revolution could occur in the not-so-distant future.




