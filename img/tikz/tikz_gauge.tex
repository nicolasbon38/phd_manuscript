
% TikZ gauge macro
\NewDocumentCommand{\verticalgauge}{O{0.3} O{5} m}{%
	% #1 = scale, #2 = height, #3 = level (0–100)
	\begin{tikzpicture}[scale=#1]
		\def\gaugewidth{1}
		\def\gaugeheight{#2}
		\pgfmathsetmacro\level{#3}
		
		% Normalized RGB [0,1]
		
		% Compute RGB manually based on piecewise logic
		\pgfmathsetmacro\rval{
			\level < 20 ? 0 :
			(\level < 80 ? (\level - 20)/60 * 100: 100)
		}
		\pgfmathsetmacro\bval{
			\level < 20 ? 100 :
			(\level < 80 ? (80 - \level)/60 * 100 : 0)
		}
		
		\pgfmathsetmacro\fillheight{\level/100 * \gaugeheight}
		
		% Draw outline
		\draw[thick] (0,0) rectangle (\gaugewidth,\gaugeheight);
		
		% Draw fill with dynamic color
		\fill[opacity=0.7, red!\rval!blue] (0,0) rectangle (\gaugewidth,\fillheight);
		
		% Optional ticks
		\foreach \y in {0,0.25,...,1} {
			\draw (-0.4,\gaugeheight*\y) -- (0,\gaugeheight*\y);
		}
		
		%		 DEBUG: print RGB values
		%		 \node at (0.5, \gaugeheight + 0.3) {\tiny R=\rval, B=\bval};
		
	\end{tikzpicture}%
}