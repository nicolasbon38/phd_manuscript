\begin{tikzpicture}[scale=0.35, transform shape]
	\definecolor{nibblecolorinner}{HTML}{00a058}
	\newcommand{\nibblecolor}{nibblecolorinner}
	
	\definecolor{bitcolorinner}{HTML}{ffa600}
	\newcommand{\bitcolor}{bitcolorinner}
	
	\definecolor{operatorcolorinner}{HTML}{b8a99a}
	\newcommand{\operatorcolor}{operatorcolorinner}
	
	\definecolor{blockcolorinner}{HTML}{ff6361}
	\newcommand{\blockcolor}{blockcolorinner}
	
	\def\bitsize{1}
	\def\bitspacing{0.2}
	\def\nibblewidth{4}
	\def\nibbleheight{1}
	\def\nibblespacing{0.8}
	
	
	\pgfmathsetmacro{\halfwidth}{\nibblewidth / 2}
	
	
	\def\metablockspacing{2}


	
	\pgfmathsetmacro{\subbyteswidth}{3 * \nibblewidth + 2 * \nibbleheight + 2 * \nibblespacing}
	\pgfmathsetmacro{\subbytesheight}{2 * \nibblewidth + 3 * \nibblespacing}
	
	\pgfmathsetmacro{\decomposerwidth}{9 * \bitsize + 6 * \bitspacing + 2 * \nibblespacing}
	\pgfmathsetmacro{\decomposerheight}{3 * \nibblewidth + 2 * \nibblespacing + \nibbleheight + \bitsize}
	
	
	\pgfmathsetmacro{\circuitsize}{11 * \bitsize + 8 * \bitspacing}
	\pgfmathsetmacro{\linearcircuitwidth}{\circuitsize + 6 * \bitsize + 2 * \bitspacing}
	\pgfmathsetmacro{\linearcircuitheight}{\circuitsize + 2 * \bitspacing}
	
		\pgfmathsetmacro{\recomposerwidth}{9 * \bitsize + 6 * \bitspacing + 2 * \nibblespacing}
	\pgfmathsetmacro{\recomposerheight}{2 * \nibblespacing + 3 * \nibblewidth + 3 * \bitsize + \nibbleheight}
	
	
	\pgfmathsetmacro{\xsubbytes}{\recomposerwidth + \metablockspacing}
	\def\ysubbytes{0}
	
		\def\xdecomposer{2 * \metablockspacing + \recomposerwidth + \linearcircuitwidth}
	\def\ydecomposer{- \metablockspacing - \subbytesheight}
	
	
		\def\xlinearcircuit{\recomposerwidth + \metablockspacing}
	\def\ylinearcircuit{- 2 * \metablockspacing - \recomposerheight - \subbytesheight}
	
		\def\xrecomposer{0}
	\def\yrecomposer{-\subbytesheight - \metablockspacing}
	
	
	
	% define the syle of teh arrows
	\tikzset{
		myarrow/.style={
    -{Latex[length=6pt, width=8pt]}, % bigger head
			thick,           % or any line width
			blue,            % or any color
			line cap=round,  % optional: smoother arrow ends
		}
	}
	
	
	 % Define the "nibble" pic
	\tikzset{
		pics/vertical_nibble/.style args={#1}{
			code={
				\node[draw, minimum width=\nibbleheight cm, minimum height=\nibblewidth cm, anchor=north west, fill=\nibblecolor, fill opacity=0.5] (#1) at (0,0) {};
			}
		}
	}
	
	\tikzset{
		pics/horizontal_nibble/.style args={#1}{
			code={
				\node[draw, minimum width=\nibblewidth cm, minimum height=\nibbleheight cm, anchor=north west, fill=\nibblecolor, fill opacity=0.5] (#1) at (0,0) {};
			}
		}
	}
	
	% Define the "bit" pic
	\tikzset{
		pics/bit/.style args={#1}{
			code={
				\node[draw, minimum width=\bitsize cm, minimum height=\bitsize cm, anchor=north west, fill=\bitcolor, fill opacity=0.5] (#1) at (0,0) {};
			}
		}
	}
	
	% Define the "bit" arith pics (I miserably failed to put it in a for loop)
	\tikzset{
		pics/bit_arith0/.style args={#1}{
			code={
				\node[draw, minimum width=\bitsize cm, minimum height=\bitsize cm, anchor=north west, fill=\nibblecolor, fill opacity=0.2] (#1) at (0,0) {};
			}
		}
	}
	
	\tikzset{
		pics/bit_arith1/.style args={#1}{
			code={
				\node[draw, minimum width=\bitsize cm, minimum height=\bitsize cm, anchor=north west, fill=\nibblecolor, fill opacity=0.4] (#1) at (0,0) {};
			}
		}
	}
	
	
	\tikzset{
		pics/bit_arith2/.style args={#1}{
			code={
				\node[draw, minimum width=\bitsize cm, minimum height=\bitsize cm, anchor=north west, fill=\nibblecolor, fill opacity=0.6] (#1) at (0,0) {};
			}
		}
	}



	\tikzset{
		pics/bit_arith3/.style args={#1}{
			code={
				\node[draw, minimum width=\bitsize cm, minimum height=\bitsize cm, anchor=north west, fill=\nibblecolor, fill opacity=0.8] (#1) at (0,0) {};
			}
		}
	}	

	
	% Define the "tree_pbs" pic
	\tikzset{
		pics/tree_pbs/.style args={#1}{
			code={
				\node[draw, minimum width=\nibblewidth cm, minimum height=\nibblewidth cm, anchor=north west, fill=\operatorcolor, fill opacity=0.5] (#1) at (0,0) {};
				
				\node[rectangle, minimum width=3cm, minimum height=1.5cm, align=center] 
				at (\nibblewidth / 2, -2 * \nibblewidth / 5) {\Huge \textbf{Tree}};
				
				\node[rectangle, minimum width=3cm, minimum height=1.5cm, align=center] 
				at (\nibblewidth / 2, -3 * \nibblewidth / 5) {\Huge \textbf{PBS}};
			}
		}
	}
	
	
	% Define the "MVB" pic
	\tikzset{
		pics/mvb/.style args={#1}{
			code={
				\node[draw, minimum width=\nibblewidth cm, minimum height=\nibblewidth cm, anchor=north west, fill=\operatorcolor, fill opacity=0.5] (#1) at (0,0) {};
				
				\node[rectangle, minimum width=3cm, minimum height=1.5cm, align=center] 
				at (\nibblewidth / 2, -\nibblewidth / 2) {\Huge \textbf{MVB}};
			}
		}
	}


	% Define the "PBS" pic
	\tikzset{
		pics/pbs/.style args={#1}{
			code={		
				\node[draw, minimum width=\nibbleheight cm, minimum height=\halfwidth cm, anchor=north west, fill=\operatorcolor, fill opacity=0.5] (#1) at (0,0) {};
				
				\node[rectangle, minimum width=1cm, minimum height=1.5cm, align=center, rotate=90] 
				at (\nibbleheight / 2, -\nibblewidth / 4) {\huge \textbf{PBS}};
			}
		}
	}
	
	
	% Define the "Sum" pic
	\tikzset{
		pics/su/.style args={#1}{
			code={			
				\node[draw, minimum width=\nibblewidth cm, minimum height=\halfwidth cm, anchor=north west, fill=\operatorcolor, fill opacity=0.5] (#1) at (0,0) {};
				
				\node[rectangle, minimum width=1cm, minimum height=1.5cm, align=center] 
				at (\nibblewidth / 2, -\nibblewidth / 4) {\scalebox{3}{$\boldsymbol{\sum}$}};
			}
		}
	}
		
	%%%%%%%%%%%%%%%%SubBytes%%%%%%%%%%%%%%%%
		
	
	 \node[draw, minimum width=\subbyteswidth cm, minimum height=\subbytesheight cm, anchor=north west] (subbytes) at (\xsubbytes,\ysubbytes) {};
	 
	 % Inputs
	\path (\xsubbytes + \nibblespacing,\ysubbytes - \nibblespacing) pic {vertical_nibble=sb_n1};
	\path (\xsubbytes + \nibblespacing,\ysubbytes - 2 * \nibblespacing - \nibblewidth) pic {vertical_nibble=sb_n2};
	  
	 % Operators
	 \path (\xsubbytes + \nibblespacing + \nibbleheight + \nibblewidth, \ysubbytes - \nibblespacing) pic {tree_pbs=sb_tree_pbs1};
	 \path (\xsubbytes + \nibblespacing + \nibbleheight + \nibblewidth, \ysubbytes -2 * \nibblespacing - \nibblewidth) pic {tree_pbs=sb_tree_pbs2};
	 
	 % Outputs
	 \path (\xsubbytes + \nibblespacing + 3 * \nibblewidth,\ysubbytes - \nibblespacing) pic {vertical_nibble=sb_n3};
	 \path (\xsubbytes + \nibblespacing + 3 * \nibblewidth,\ysubbytes - 2 * \nibblespacing - \nibblewidth) pic {vertical_nibble=sb_n4};
	 
	 \draw[myarrow](sb_n1.east) -- (sb_tree_pbs1.west);
	 \draw[myarrow](sb_n1.east) -- (sb_tree_pbs2.west);
	 \draw[myarrow](sb_n2.east) -- (sb_tree_pbs1.west);
	\draw[myarrow](sb_n2.east) -- (sb_tree_pbs2.west);
	
	\draw[myarrow](sb_tree_pbs1.east) -- (sb_n3.west);
	\draw[myarrow](sb_tree_pbs2.east) -- (sb_n4.west);
	
	
	%%%%%%%%%%%%%%%%Decomposer%%%%%%%%%%%%%%%%

	\node[draw, minimum width=\decomposerwidth cm, minimum height=\decomposerheight cm, anchor=north west] (decomposer) at (\xdecomposer,\ydecomposer) {};
	
	%%Input
	\path (\xdecomposer + \nibblespacing + \bitspacing, \ydecomposer - \nibblespacing) pic {horizontal_nibble=dec_n1};
	\path (\xdecomposer + \nibblespacing + 5 * \bitspacing + \nibbleheight + \nibblewidth, \ydecomposer - \nibblespacing) pic {horizontal_nibble=dec_n2};
	
	% Operators
	\path (\xdecomposer + \nibblespacing + \bitspacing, \ydecomposer - \nibblespacing - \nibblewidth) pic {mvb=mvb1};
	\path (\xdecomposer + \nibblespacing + 5 * \bitspacing + \nibbleheight + \nibblewidth, \ydecomposer - \nibblespacing - \nibblewidth) pic {mvb=mvb2};
	
	\draw[myarrow] (dec_n1.south) -- (mvb1.north);
	\draw[myarrow] (dec_n2.south) -- (mvb2.north);
	
	% Outputs
	\foreach \i in {0,...,3} {
		\path 
		({\xdecomposer + \nibblespacing + \i * (\bitspacing + \bitsize)}, 
		{\ydecomposer - \nibblespacing - \nibbleheight - 3 * \nibblewidth}) 
		pic {bit=dec_bit\i};
		
		\path 
		({\xdecomposer + \nibblespacing + \bitsize + (4 + \i) * (\bitspacing + \bitsize)}, 
		{\ydecomposer - \nibblespacing - \nibbleheight - 3 * \nibblewidth}) 
		pic {bit=dec_bit{\i+4}};
		
		\draw[myarrow] (mvb1.south) -- (dec_bit\i.north);
		\draw[myarrow] (mvb2.south) -- (dec_bit{\i+4}.north);
	}
	
	
	
	%%%%%%%%%%%%%%%%Linear Circuit%%%%%%%%%%%%%%%%

	\node[draw, minimum width=\linearcircuitwidth cm, minimum height=\linearcircuitheight cm, anchor=north west] (linearcircuit) at (\xlinearcircuit, \ylinearcircuit) {};
	
	\node at (\xlinearcircuit + \linearcircuitwidth / 2, \ylinearcircuit - \linearcircuitheight / 2 + \linearcircuitheight / 20) {\Huge \textbf{Linear Circuit}};
	\node at (\xlinearcircuit + \linearcircuitwidth / 2, \ylinearcircuit - \linearcircuitheight / 2 - \linearcircuitheight / 20) {\scalebox{3}{$\mathcal C$}};
	
	% Outputs
	\foreach \i in {0,...,3} {
		\path 
		({\xlinearcircuit + \nibblespacing}, 
		{\ylinearcircuit- 2 * \bitspacing - \i * (\bitspacing + \bitsize) }) 
		pic {bit=lin_bit\i};
		
		\path 
		({\xlinearcircuit + \nibblespacing}, 
		{\ylinearcircuit- \bitspacing - (\i+4) * (\bitspacing + \bitsize) - \bitsize}) 
		pic {bit=lin_bit{\i + 4}};
		
		\path 
		({\xlinearcircuit + \linearcircuitwidth - \nibblespacing - \bitsize}, 
		{\ylinearcircuit- 2 * \bitspacing - \i * (\bitspacing + \bitsize) }) 
		pic {bit=lin_bit{\i + 8}};
		
		
		\path 
		({\xlinearcircuit + \linearcircuitwidth - \nibblespacing - \bitsize}, 
		{\ylinearcircuit- \bitspacing - (\i+4) * (\bitspacing + \bitsize) - \bitsize}) 
		pic {bit=lin_bit{\i + 12}};
		

		\draw[myarrow] (lin_bit{\i + 8}) -- ++(-\nibblewidth, 0);
		\draw[myarrow] (lin_bit{\i + 12}) -- ++(-\nibblewidth, 0);
		
		\draw[myarrow] ($ (lin_bit\i) + (\nibblewidth, 0) $) -- (lin_bit\i);
		\draw[myarrow] ($ (lin_bit{\i + 4}) + (\nibblewidth, 0) $) -- (lin_bit{\i + 4});
	}
	
	\foreach \i in {0,..., 7}{
		% Round key
		\path 
		({\xlinearcircuit + 2 * \nibblespacing + 2 * \nibbleheight + \bitsize + \i * (\bitsize + \bitspacing)}, 
		{\ylinearcircuit + 2 * \bitsize + \bitspacing}) 
		pic {bit=lin_bit{\i + 16}};	
		
		\draw[myarrow] (lin_bit{\i + 16}) -- ++(0, -2*\bitsize - \bitspacing);
	}
	
	\draw ({\xlinearcircuit + \nibblespacing + 4 * \bitsize}, {\ylinearcircuit - \bitspacing}) -- ++(-\bitsize, 0) -- ++(0, -\circuitsize) -- ++(\nibblespacing, 0);
	
	\draw ({\xlinearcircuit + \linearcircuitwidth - 4 * \bitsize - \nibblespacing}, {\ylinearcircuit - \bitspacing}) -- ++(\bitsize, 0) -- ++(0, -\circuitsize) -- ++(-\nibblespacing, 0);
	
	
	\node at ({\xlinearcircuit + 2 * \nibblespacing + 2 * \nibbleheight + \bitsize + 4 * (\bitsize + \bitspacing) }, 
	{\ylinearcircuit + 3 * \bitsize + \bitspacing}) 
	{\Huge \textbf{Round Key}};	
	
	
	%%%%%%%%%%%%%%%%%%%%%%%%%Recomposer%%%%%%%%%%%%%%%%%%%%%%%%%%
	

	\node[draw, minimum width=\recomposerwidth cm, minimum height=\recomposerheight cm, anchor=north west] (recomposer) at (\xrecomposer, \yrecomposer) {};
	
	
	\foreach \i in {0,...,3} {
		% Input	
		\path 
		({\xrecomposer + \nibblespacing + \i * (\bitspacing + \bitsize)}, 
		{\yrecomposer - \recomposerheight +\nibblespacing + \bitsize}) 
		pic {bit=rec_bit\i};
		
		\path 
		({\xrecomposer + \nibblespacing + (\i + 4) * (\bitspacing + \bitsize) + \bitsize}, 
		{\yrecomposer - \recomposerheight + \nibblespacing + \bitsize}) 
		pic {bit=rec_bit{\i + 4}};
		
		
		% PBS layer
		\path 
		({\xrecomposer + \nibblespacing + \i * (\bitspacing + \bitsize)}, 
		{\yrecomposer - \recomposerheight + \nibblespacing + \bitsize + \nibblewidth}) 
		pic {pbs=rec_pbs\i};
		
		
		\path 
		({\xrecomposer + \nibblespacing + (\i + 4) * (\bitspacing + \bitsize) + \bitsize}, 
		{\yrecomposer - \recomposerheight + \nibblespacing + \bitsize + \nibblewidth}) 
		pic {pbs=rec_pbs{\i + 4}};
		
		\draw[myarrow] (rec_bit\i) -- (rec_pbs\i);
		\draw[myarrow] (rec_bit{\i + 4}) -- (rec_pbs{\i + 4});
		
		% bit arith layer
		\path 
		({\xrecomposer + \nibblespacing + \i * (\bitspacing + \bitsize)}, 
		{\yrecomposer - \recomposerheight + \nibblespacing + \bitsize + 1.5 * \nibblewidth + \bitsize}) 
		pic {bit_arith\i=bit_arith\i1};
		
		\path 
		({\xrecomposer + \nibblespacing + (\i + 4) * (\bitspacing + \bitsize) + \bitsize}, 
		{\yrecomposer - \recomposerheight + \nibblespacing + \bitsize + 1.5 * \nibblewidth + \bitsize}) 
		pic {bit_arith\i=bit_arith\i2};
		
		\draw[myarrow] (rec_pbs\i) -- (bit_arith\i1);
		\draw[myarrow] (rec_pbs{\i + 4}) -- (bit_arith\i2);
	}	
	
	
	\path 
	({\xrecomposer + \nibblespacing + \bitspacing}, 
	{\yrecomposer - \recomposerheight + \nibblespacing + \bitsize + 3 * \nibblewidth}) 
	pic {su=su1};
	
	\path 
	({\xrecomposer + \recomposerwidth - \nibblespacing - \bitspacing - \nibblewidth}, 
	{\yrecomposer - \recomposerheight + \nibblespacing + \bitsize + 3 * \nibblewidth}) 
	pic {su=su2};
	
	\foreach \i in {0,...,3}{
		\draw[myarrow]	(bit_arith\i1) -- (su1);
		\draw[myarrow]	(bit_arith\i2) -- (su2);
	}
	
	
	%output
	\path ({\xrecomposer + \nibblespacing + \bitspacing}, 
	{\yrecomposer - \recomposerheight + \nibblespacing + \bitsize + 3 * \nibblewidth + \nibbleheight + \halfwidth})  pic {horizontal_nibble=rec_n1};
	
	\path ({\xrecomposer + \recomposerwidth - \nibblespacing - \bitspacing - \nibblewidth}, 
	{\yrecomposer - \recomposerheight + \nibblespacing + \bitsize + 3 * \nibblewidth+ \nibbleheight + \halfwidth})  pic {horizontal_nibble=rec_n2};
	
	\draw[myarrow] (su1) -- (rec_n1);
	\draw[myarrow] (su2) -- (rec_n2);
	
	
	
	\draw[myarrow] (subbytes) -| (decomposer);
	\draw[myarrow] (decomposer) |- (linearcircuit);
	\draw[myarrow] (linearcircuit) -| (recomposer);
	\draw[myarrow] (recomposer) |- (subbytes);
	
	%gauges
	\node at ($(linearcircuit.west) + (-2 * \metablockspacing - \recomposerwidth / 2, 0)$) {\scalebox{3}{\verticalgauge{90}}};
	
	\node at ($(linearcircuit.east) + (2 * \metablockspacing + \recomposerwidth / 2, 0)$) {\scalebox{3}{\verticalgauge{40}}};
	
	\node at ($(subbytes.east) + (2 * \metablockspacing + \recomposerwidth / 2, 2 * \metablockspacing)$) {\scalebox{3}{\verticalgauge{20}}};
	
	\node at ($(subbytes.west) + (-2 * \metablockspacing - \recomposerwidth / 2, 2 * \metablockspacing)$) {\scalebox{3}{\verticalgauge{40}}};
	
	\node at ($(recomposer.east) + (\metablockspacing, 0)$) {\scalebox{3}{\verticalgauge{20}}};
	
	
	% Titles
	\node at ($(subbytes.north) + (0, \metablockspacing / 2)$) {\Huge \textbf{SubBytes}};
	\node[rotate=-90] at ($(decomposer.east) + (\metablockspacing / 2, 0)$) {\Huge \textbf{Decomposer}};
	\node[rotate=90] at ($(recomposer.west) + (-\metablockspacing / 2, 0)$) {\Huge \textbf{Recomposer}};
	
\end{tikzpicture}